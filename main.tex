\documentclass{tesi}



\title{Virtualized meeting room: realtà virtuale per smart working e formazione}
\author{Lorenzo Goldoni}
\matricola{456802}
\relatore{Prof. Maurizio Patrignani}
\correlatore{Raphaël Bussa}
\annoAccademico{2016/2017}

\dedica{} % solo se nel documento si usa il comando \generaDedica
% --- FINE dati relativi al template TesiDiaUniroma3

% --- INIZIO richiamo di pacchetti di utilità. Questi sono un esempio e non sono strettamente necessari al modello per la tesi.
\usepackage[plainpages=false]{hyperref}	% generazione di collegamenti ipertestuali su indice e riferimenti
\usepackage[all]{hypcap} % per far si che i link ipertestuali alle immagini puntino all'inizio delle stesse e non alla didascalia sottostante
\usepackage{amsthm}	% per definizioni e teoremi
\usepackage{amsmath}
\usepackage{float}
\usepackage{wrapfig}

\begin{document}
	
\frontmatter
\generaFrontespizio
\generaDedica
\ringraziamenti{Ringraziamenti}	% inserisce i ringraziamenti e li prende in questo caso da ringraziamenti.tex
\generaIndice
%\generaIndiceFigure
\introduzione{Introduzione}		% inserisce l'introduzione e la prende in questo caso da introduzione.tex

% ----- Pagine di tesi, numerate in arabo (1,2,3,4,...) (obbligatorio)
\mainmatter
% il comando ``capitolo'' ha come parametri:
% 1) il titolo del capitolo
% 2) il nome del file tex (senza estensione) che contiene il capitolo. Tale nome \`e usato anche come label del capitolo
\capitolo{Realtà virtuale}{capitolo1}
\capitolo{Smart working e ambienti di lavoro virtualizzati}{capitolo2}
\capitolo{Requisiti del progetto}{capitolo3}
\capitolo{Tecnologie adottate}{capitolo4}
\capitolo{Analisi del progetto}{capitolo5}
% ----- Parte finale della tesi (obbligatorio)
\backmatter
\conclusioni{conclusioni}

% Bibliografia con BibTeX (obbligatoria)
% Non si deve specificare lo stile della bibliografia
\bibliography{bibliografia} % inserisce la bibliografia e la prende in questo caso da bibliografia.bib
\end{document}