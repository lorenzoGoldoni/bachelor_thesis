% !TeX spellcheck = it_IT
Questo capitolo tratta dell'analisi dell'applicazione dal punto di vista della raccolta dei requisiti derivante dall'incontro con il cliente a cui va destinata. \\

\section{Una Sala Riunioni in Realtà Virtuale}
L'applicazione sviluppata e descritta in questa tesi è finalizzata alla creazione di sale riunioni virtuali a cui più utenti possono accedere contemporaneamente con l'utilizzo di un Samsung Gear VR. L'applicazione è pensata su un modello enterprise, ovvero il suo target principale non saranno i consumatori, ma grosse aziende che vogliono facilitare le riunioni tra dipendenti o gli incontri con i clienti. La creazione di una sala riunioni virtuale introdurrebbe i benefici dello smart working nei contesti aziendali. Lo scopo non è quello di sostituire gli attuali strumenti di teleconferenza (Skype, Google Hangouts, ecc.), ma di ampliarne il concetto, fornendo nuovi modelli e iniziando un processo di aggiornamento tecnologico.\\
Essendo il cliente lo sviluppatore e il produttore dei visori, vuole che l'applicazione funga da fattore chiave nella scelta dei loro dispositivi da parte di grandi aziende, andando ad ampliare le loro possibilità con nuovi strumenti e facendoli sentire parte della rivoluzione tecnologica. \\

\section{Principali Funzionalità e Casi d'Uso}
L'applicazione finale dovrà contenere molte funzionalità, andando ad imitare e ampliando gli strumenti di una comune sala riunione. Una volta avviata l'applicazione, si troverà in un'ambiente virtuale dal quale, attraverso l'interazione del controller con i menu posti su pannelli nel mondo virtuale, l'utente potrà accedere al suo account personale. Una volta effettuato l'accesso, potrà personalizzare il suo profilo (avatar,nome,notifiche,ecc), creare una riunione con l'apposita sala e invitare altri utenti a parteciparvi. All'interno di quest'ultima ogni utente sarà rappresentato da un'avatar virtuale che seguirà i reali movimenti della testa. Per tutta la durata della conferenza sarà attiva la chat vocale che permetterà una comunicazione rapida e naturale tra i partecipanti.  Nella sala riunioni avranno la possibilità di visualizzare documenti, sia personalmente che in condivisione su un proiettore, modificarli sottolineandoli o evidenziandone le parti importanti e di condividerli tra i partecipanti alla riunione. Di questa parte è particolarmente importante la gestione sicura dei dati che transiteranno sull'applicazione e sui server che ne gestiscono la parte in rete, poiché essendo il target principale grande corporazioni, la sensibilità dei documenti è elevata. Sebbene molte di queste funzioni non sono ancora implementate, è importante elencarle perché sono state tenute in considerazione e hanno condizionato lo sviluppo della versione dimostrativa sviluppata durante questo tirocinio.\\

Di seguito il  caso d'uso dell'applicazione dimostrativa:
\begin{enumerate}
	\item l'utente avvia l'applicazione e inserisce il dispositivo nel visore;
	\item l'utente si ritrova in un ambiente virtualizzato;
	\item l'utente sceglie tra il suo avatar tra un modello maschile e uno femminile;
	\item l'utente inserisce il nome che verrà visualizzato sopra il suo avatar nell'effettiva sala riunioni con l'ausilio di una tastiera virtuale;
	\item l'utente conferma il nome inserito;
	\item l'utente avvia una riunione;
	\item il sistema si collega ad un server creando una stanza pubblica; 
	\item il sistema si collega ad un server per gestire la chat vocale;
	\item l'utente si troverà in una sala riunioni virtualizzata assieme ad eventuali altri utenti, che hanno compiuto i passi 1-8,  e ne visualizzerà l'avatar;
	\item attraverso un menu posto di fronte all'utente, quest'ultimo potrà visualizzare i documenti presenti sul cellulare e proiettarli sullo schermo posto nella sala riunioni;
	\item l'utente può disconnettersi dalla sala riunioni e ritornare alla stanza iniziale;
	\item il sistema chiude la connessione con i server;
\end{enumerate}

\section{Esperienza d'uso}
Essendo un'applicazione che verrà utilizzata in contesti lavorativi o di formazione, è fondamentale ottimizzare l'esperienza d'uso dell'utente. Grande attenzione infatti è stata riposta nella fluidità con la quale l'applicazione traccia i movimenti della testa. Quest'ultimo fattore infatti è ciò che più contraddistingue questo strumento da gli altri di teleconferenza, nei quali un mancato approccio visivo tra i partecipanti può creare confusione, rendendo difficile seguire o intervenire in un discorso. Anche la qualità dei documenti gioca un ruolo fondamentale, poiché la visualizzazione e la loro condivisione deve essere il più naturale possibile ricreando la sensazione di una vera riunione. Lo stesso discorso si può applicare alla chat vocale che gioca il ruolo più importante nella comunicazione all'interno della stanza.
