Negli ultimi anni la realtà virtuale ha aumentato la sua popolarità grazie agli avanzamenti tecnologici ottenuti nel campo dei telefoni cellulari. Stiamo ancora muovendo i primi passi in questa nuova tecnologia, ma le sue potenzialità sono già evidenti. Inizialmente infatti i sistemi immersivi erano esclusivamente utilizzabili dai ricercatori, ma oggi vi è una distribuzione su larga scala e la realtà virtuale viene impiegata nei contesti più disparati. Questo nuovo mercato ha catturato l'attenzione dei "giganti" del settore informatico che ora investono risorse significative su dispositivi o applicativi per la realtà virtuale.
Lo sviluppo di un'applicazione immersiva è pero un processo ancora complesso, poiché oltre alla scarsa documentazione, bisogna tenere in conto fattori sinora fuori dalle competenze di uno sviluppatore, come ad esempio la salute dell'utente. Per questo molti programmatori hanno iniziato a sviluppare librerie o motori grafici che ne semplificassero lo sviluppo, così da poter lasciare lo sviluppatore più libero di concentrarsi sui contenuti. Uno di questi strumenti è il motore grafico Unity che permette la creazione di progetti in realtà virtuale, compatibili con diverse piattaforme. \\
\textit{Virtualized Meeting Room} è un'applicazione mobile in realtà virtuale per la gestione in rete di sale riunioni virtuali. In quest'ultima l'utente è rappresentato da un avatar e ha la possibilità di parlare con altri utenti connessi alla stessa stanza e di inviare loro dei documenti presenti sul telefono. L'applicazione è pensata per funzionare con i visori Samsung Gear VR. Lo scopo principale di questo progetto è quello di facilitare l'attuazione dei concetti di smart working e di rendere più efficaci le tecniche di formazione aziendali. \\
\newpage
L'attività di tirocinio è stata svolta presso l'azienda Fingerlinks S.r.l. per la quale lavoro e sono stato affiancato dal tutor Raphaël Bussa.
Durante l'attività di tirocinio mi sono occupato dell'analisi e dello sviluppo dell'intera applicazione, in particolare di:
\begin{enumerate}
	\item Incontrare il cliente
	\item Stilare i casi d'uso
	\item Designare l'esperienza virtuale
	\item Progettare e ottimizzare l'ambiente grafico della simulazione
	\item Creare le interfacce grafiche
	\item Progettare e sviluppare le classi software
	\item Connettere e far interagire in rete due Gear VR
	\item Gestire lo scambio di dati tra gli utenti
\end{enumerate}

Nel primo capitolo si descrive la realtà virtuale partendo dalla sua definizione e arrivando alle attuali tecnologie che la rendono accessibile, spiegando in breve le componenti hardware e software.\\
\\
Il secondo capitolo tratta il concetto di smart working e i benefici che può portare agli ambienti di lavoro se unito alla realtà virtuale. Vengono discussi anche i limiti dell'attuale stato dell'arte.\\
\\
Nel terzo capitolo si descrivono le tecnologie utilizzate durante lo sviluppo dell'applicazione Virtualized Meeting Room, ovvero Unity, con le proprie risorse e il visore GearVR. Viene anche discusso l'ambiente di sviluppo e l'architettura software del motore grafico.\\
\\
Il quarto capitolo tratta dell'analisi dell'applicazione dal punto di vista della raccolta dei requisiti imposti dal cliente a cui va destinato l'applicativo.\\
\\
Nel quarto capitolo si descrive la progettazione e lo sviluppo dell'applicazione, trattando degli elementi chiave e le classi software più significative.  

