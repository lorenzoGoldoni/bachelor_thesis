L'obbiettivo di questo tirocinio è stato quello di creare un'applicazione compatibile con i visori Gear VR che permettesse a più utenti di interagire, parlare e condividere documenti in un unico ambiente virtuale.\\

Durante la fase di sviluppo e la progettazione ho acquisito delle competenze tra le quali la conoscenza dei processi d'ottimizzazione di un'applicazione VR all'interno del motore grafico Unity, le implementazioni di librerie che gestiscono la parte relativa al \textit{networking} di un'applicazione in realtà virtuale e il funzionamento di diverse classi native Android. \\

Il tirocino è avvenuto presso l'azienda Fingerlinks S.r.l. per la quale lavoro. Mi sono occupato dell'intera progettazione e sviluppo dell'applicazione, imparando ad interfacciarmi con un cliente di grande prestigio.\\

Per quanto riguarda gli sviluppi futuri, oltre all'implementazione delle funzioni già richieste dall'utente e descritte nel capitolo quattro, si potrebbe pensare di integrare la libreria Samsung \textit{Knox} per aumentare la sicurezza nella gestione dati all'interno dell'applicazione. Inoltre il prodotto finale dovrebbe essere accompagnato da un'altra applicazione nativa che gestisca gli inviti alle riunioni e la selezione dei documenti da condividere, così da snellire la struttura dell'applicazione in realtà virtuale lasciando l'utente più libero di immergersi nella simulazione.

Concludo affermando che, in base alle aspettative circa lo sviluppo di questo software, gli obbiettivi prefissati sono stati ampiamente soddisfatti.